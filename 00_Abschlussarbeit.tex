%%%%%%%%%%%%%%%%%%%%%%%%%%%%%%%%%%%%%%%%%%%%%%%%%%%%%%%%%%%%%%%%%%%%%%%%%%%%%%%%%%%%%%%%%%%%%%%%%%%%%%%%%%%%%%%%%%%%
%
% Masterarbeit im Fachbereich 1
% Systems Engineering - HTW Berlin
% 
% Thema:
% Lorem Ipsum
%
% \newcommand{\draftdate}{140910}
% \newcommand{\rev}{108}
% %
\newcommand{\thesisopening}{01. Dezember 2014}
\newcommand{\thesisclosing}{09. Februar 2015}
\newcommand{\thesisrelease}{09. Februar 2015}
%%%%%%%%%%%%%%%%%%%%%%%%%%%%%%%%%%%%%%%%%%%%%%%%%%%%%%%%%%%%%%%%%%%%%%%%%%%%%%%%%%%%%%%%%%%%%%%%%%%%%%%%%%%%%%%%%%%%
%%%
%%
%
%%%%%%%%%%%%%%%%%%%%%%%%%%%%%%%%%%%%%%%%%%%%%%%%%%%%%%%%%%%%%%%%%%%%%%%%%%%%%%%%%%%%%%%%%%%%%%%%%%%%%%%%%%%%%%%%%%%%
%%%%%%%%%%%%%%%%%%%%%%%%%%%%%%%% D O K U M E N T K O N F I G U R A T I O N E N %%%%%%%%%%%%%%%%%%%%%%%%%%%%%%%%%%%%%
%%%%%%%%%%%%%%%%%%%%%%%%%%%%%%%%%%%%%%%%%%%%%%%%%%%%%%%%%%%%%%%%%%%%%%%%%%%%%%%%%%%%%%%%%%%%%%%%%%%%%%%%%%%%%%%%%%%%
%
%%
%%% 
% Dokumentformatierung
\documentclass
[a4paper,german,
11pt												% ersatzweise 12pt, wenn mehr Seiten entstehen sollen
]
{scrreprt}
%%%
%
%%%%%%%%%%%%%%%%%%%%%%%%%%%%%%%%%%%%%% P A C K A G E S %%%%%%%%%%%%%%%%%%%%%%%%%%%%%%%%%%%%%%%%%
%
%%% Textformatierung %%%
%
\usepackage[onehalfspacing]{setspace}				% Zeilenabstand bestimmen
\usepackage[table]{xcolor}							% farbiger Text
\usepackage{colortbl}								% Tabellen mit Hintergrundfarben versehen
\usepackage{lettrine}								% Kapitelbeginn mit großem Buchstaben
\usepackage{fancybox}								% Schattierungen und Rahmen für Grafiken
\usepackage{wrapfig}								% Paket zur Positionierung von Grafiken
\usepackage{capt-of}								% Untertitel von Abbildungen, Tabellen usw. in Gleitumgebung
\usepackage[ngerman]{babel,translator}				% Silbentrennung nach der neuen deutschen Rechtschreibung, z.B.: Sys-tem
\usepackage{paralist}								% Kompakte Aufzaehlungen erstellen
\usepackage[bottom]{footmisc}							% Fussnoten am Seitenende
\usepackage{fancyhdr}								% Kopf- und Fußzeilenformatierung
\usepackage{pdfpages}								% für die Einbindung kompletter pdf-*Seiten*
\usepackage{eso-pic}								% Wasserzeichen z.B. "`ENTWURF"' auf .pdf erzeugen
\usepackage[hyphens]{url}							% für \url{http://www}, Option hyp erlaubt auch Umbruch nach "-"
\usepackage[numbers,square]{natbib}					% Für \setlength{\bibsep}{3mm}; square macht eckige Klammern
\usepackage{bibgerm}								% Styles für Literaturverzeichnisse
\usepackage{titlesec, blindtext, color}				% Styles für Kapitelüberschriften
\usepackage[printonlyused,withpage]{acronym}		% Abkürzungsverzeichnis
%
\usepackage[utf8]{inputenc}							% Zeichensatz, ermöglicht die direkte Eingabe von Umlauten im Editor
\usepackage[T1]{fontenc}							% Optimiert Umlautverwendung
\usepackage{lmodern}								% Schriften aufgrund von 'fontenc' glätten
\usepackage{graphicx}								% Einbindung von Grafiken (pdf, png, jpg)
\usepackage{float}									% bietet Option [H] für bombenfestes Verankern
\usepackage{enumerate}								% verbessert Aufzählungen
\usepackage{expdlist}								% erweiterter Funktionssatz für description
\usepackage{chngcntr}								% Konfiguration des Fußnotenzählers ermöglichen
\usepackage{array}									% für Tabellen: bindet tabular-Umgebung ein
\usepackage{parskip}								% zw. Absätzen: eine knappe Leerzeile statt hängender Einzüge
\usepackage[right]{eurosym}							% Eurosymbol
\usepackage{makeidx}								% Package zur Indexerstellung
\usepackage{multicol}								% zur Indexerstellung in zwei Spalten
\usepackage{multirow}								% Verbinden von Zellen in Tabellen
\usepackage{booktabs}								% Tabellenlayout anpassen
\usepackage{longtable}								% Tabellen über mehrere Seiten strecken
\usepackage{chngcntr}								% Nummeriegung der Abbildungen beeinflussen (z.B. nach Kapitel/fortlaufend)
\usepackage[outercaption]{sidecap}					% Bildunterschriften rechts neben der Abbildung positionieren
\usepackage[bf]{caption}							% Kuerzel und Nummerierung von Bildunterschriften "`fett"' schreiben
\usepackage{amssymb}								% Erweiterte Symboldarstellung
%%
% Packagehinweise
% Folgende Packete müssen zusätzlich von MiKTeX bereitgestellt werden:
% setspace, xcolor, colortbl, lettrine, fancybox, wrapfig, capt-of, translator, paralist, expdlist, footmisc, fancyhdr, pdfpages, eso-pic, url,
% natbib, bibgerm, titlesec, blindtext, acronym, suffix, xstring, glossaries, etoolbox, textcase, xfor, datatool-base, substr, fp, supertabular,
% mptopdf
%%
%%%%%%%%%% NEW FOR TESTING %%%%%%%%%%
%
%%
%%% Abstellgleis %%%
%
%Options: Sonny, Lenny, Glenn, Conny, Rejne, Bjarne, Bjornstrup
%\usepackage[Glenn]{fncychap}
%%
%%% Referenzierung / Dokumentenlinks %%%
% LINK packages IMMER zuletzt einbinden!!!
\usepackage{hyperref}			% Package für Textlinks in pdf-Datei / Option [colorlinks] färbt Text rot!
\usepackage[toc,nonumberlist,nopostdot]{glossaries}	% Automatisches Glossar erstellen
%%%
%
%%%%%%%%%%%%%%%%%%%%%%%%%%%%%% W A S S E R Z E I C H E N %%%%%%%%%%%%%%%%%%%%%%%%%%%%%%%%%%%%%%%
%
\AddToShipoutPicture{%
    \AtTextCenter{%
      \makebox(0,0)[c]{\resizebox{\textwidth}{!}{%
        \rotatebox{45}{\textsf{\textbf{\color{lightgray}DRAFT//\draftdate}}}}} 
    }
  }
\ClearShipoutPicture
%%%
%
%%%%%%%%%%%%%%%%%%%%%%%%%%%%%%%%%%%%%%%% D E F I N E %%%%%%%%%%%%%%%%%%%%%%%%%%%%%%%%%%%%%%%%%%%
%
\definecolor{darkred}{rgb}{0.7,0.0,0.0}				% Textfarben definieren
\definecolor{hellgrau}{rgb}{0.95,0.95,0.95}
\definecolor{dunkelgrau}{rgb}{0.8,0.8,0.8}
\sloppy												% großzügiger Zeilenumbruch -> keine rechts rausragenden Zeilen mehr
%
\def\TReg{\textsuperscript{\textregistered}}
\def\TCop{\textsuperscript{\textcopyright}}
\def\TTra{\textsuperscript{\texttrademark}}
%
\newcommand{\ctab}{\centering\arraybackslash}		% Text in Tabellenzellen zentrieren
\newcolumntype{C}[1]{>{\centering\arraybackslash}p{#1}} % Text spaltenweise zentrieren
%
% Abstand zwischen Nummerierung und Text veraendern
\makeatletter
\renewcommand*\l@figure{\@dottedtocline{1}{1.5em}{3em}}
\renewcommand*\l@table{\@dottedtocline{1}{1.5em}{3em}}
\makeatother
%
% Farben fuer Verknuepfungen/Verweise anpassen
\hypersetup{
	linktocpage			= true,
	colorlinks			= true,
	linkcolor			= black,
	citecolor			= green,
	filecolor			= magenta,
	urlcolor			= cyan,
	pdftitle			= {Thesis M. Musterfrau},
	pdfauthor			= Monika Musterfrau,
	pdfsubject			= {Untersuchungen zum Vernetzen der Netze},
	pdfdisplaydoctitle	= true,
%	bookmarks			= true,
	bookmarksnumbered	= true,
	pdfpagemode			= UseOutlines,
	pdfpagelayout		= SinglePage,
	pdfprintscaling		= None,
}
%%%
%
%%%%%%%%%%%%%%%%%%%%%%%%%%%%%%%%%%%%% F O O T N O T E S %%%%%%%%%%%%%%%%%%%%%%%%%%%%%%%%%%%%%%%%%
% Fortlaufende Fußnoten (regulär wird der Zähler zu Beginn eines neuen Kapitels zurückgesetzt)
%
\counterwithout{footnote}{chapter}
%%%
%
%%%%%%%%%%%%%%%%%%%%%%%%%%%%%%%%%%%%%% C A P T I O N S %%%%%%%%%%%%%%%%%%%%%%%%%%%%%%%%%%%%%%%%%%
%
% Beschriftung für Abbildungen und Tabellen formatieren
\addto\captionsngerman{\renewcommand\figurename{Abb.}}
\addto\captionsngerman{\renewcommand\tablename{Tab.}}
\counterwithin{figure}{section}
\counterwithin{table}{section}
%%%
%
%%%%%%%%%%%%%%%%%%%%%%%%% L I T E R A T U R V E R Z E I C H N I S %%%%%%%%%%%%%%%%%%%%%%%%%%%%%%
%
% Literaturverzeichnis mit BibTeX
%
\bibliographystyle{alphadin}						% 
\setlength{\bibsep}{3mm}							% Abstände im Literaturverzeichnis
%%%
%
%%%%%%%%%%%%%%%%%%%%%%%%%%%%%%% K O P F - & F U ß Z E I L E %%%%%%%%%%%%%%%%%%%%%%%%%%%%%%%%%%%%
%
% Größenanpassungen
%
\setlength{\headheight}{16pt}						% Höhe der Kopfzeile anpassen
\setlength{\unitlength}{1cm}
\setlength{\oddsidemargin}{0.3cm}
\setlength{\evensidemargin}{0.3cm}
\setlength{\textwidth}{16.5cm}
\setlength{\topmargin}{-1.2cm}
\setlength{\textheight}{24.5cm}
\columnsep 0.5cm
%%%
%
%%%%%%%%%%%%%%%%%%%%%%%%%%%% I N H A L T S V E R Z E I C H N I S %%%%%%%%%%%%%%%%%%%%%%%%%%%%%%%
%
% Strukturtiefe des Inhaltsverzeichnis
%
\setcounter{secnumdepth}{5}
\setcounter{tocdepth}{5}
%%%
%
%%%%%%%%%%%%%%%%%%%%%%%%%%%%%% T R E N N U N G S R E G E L N %%%%%%%%%%%%%%%%%%%%%%%%%%%%%%%%%%%
%
% Anmerkung: für Wörter mit Umlauten muss das Paket \usepackage[T1]{fontenc} eingebunden werden --
% in der vorliegenden Version funktionieren *keine* Umlaute!!
% 
\hyphenation{Samm-lung-en Samm-lung Stau-beck-en Vor-na-me-in-i-ti-al Ver-stär-ker-aus-gang Nach-na-me Kurz-be-zeich-nung deutsch-spra-chige
deutsch-sprachig Screen-shot Screen-shots schluss-end-lich Schluss-end-lich Make-In-dex Da-tei-name Da-tei-namen Ur-instinkt Ur-instinkte
Lea-sing-part-ner  Va-li-da-tion In-fra-struk-tur-en Ver-ant-wort-lich-keit
exis-tie-ren-der Ver-si-on-en Netz-werk-la-tenz-en Re-fe-renz-um-ge-bung Im-ple-men-tie-rung}
%%%
%
%%%%%%%%%%%%%%%%%%%%%%%%%%%%%%%%%%% G L O S S A R %%%%%%%%%%%%%%%%%%%%%%%%%%%%%%%%%%%%%%%%%%%%%%
%
%\loadglsentries{glossary/glossary}					% Externes Glossarverzeichnis laden
\makeglossaries										% Aufruf zur automatischen Erstellung des Glossars
%%%
%
%%%%%%%%%%%%%%%%%%%%%%%%% T E X T F L U S S - P A R A M E T E R %%%%%%%%%%%%%%%%%%%%%%%%%%%%%%%%
%
\clubpenalty = 10000 
\widowpenalty = 10000
\displaywidowpenalty = 10000
\interlinepenalty = 5000							% u.a. sauberen Umbruch bei Eintraegen in Verzeichnissen!
%%%
%%
%
%%%%%%%%%%%%%%%%%%%%%%%%%%%%%%%%%%%%%%%%%%%%%%%%%%%%%%%%%%%%%%%%%%%%%%%%%%%%%%%%%%%%%%%%%%%%%%%%%%%%%%%%%%%%%%%%%%%%
%%%%%%%%%%%%%%%%%%%%%%%%%%%%%%%%%%%%%% D O K U M E N T E R S T E L L U N G %%%%%%%%%%%%%%%%%%%%%%%%%%%%%%%%%%%%%%%%%
%%%%%%%%%%%%%%%%%%%%%%%%%%%%%%%%%%%%%%%%%%%%%%%%%%%%%%%%%%%%%%%%%%%%%%%%%%%%%%%%%%%%%%%%%%%%%%%%%%%%%%%%%%%%%%%%%%%%
%
%%
%%%
\begin{document}
%%%
%
%%%%%%%%%%%%%%%%%%%%%%%%%%%%%%% I N L I N E T I T L E P A G E %%%%%%%%%%%%%%%%%%%%%%%%%%%%%%%%%%
% Dokumentinformationen zur automatischen Erstellung einer Titelseite
% Ist nur erforderlich, wenn keine eigenständige Titelseite eingebunden wird!

%\title{TEST}										% Legt den Titel des Dokuments fest
%\author{M. Musterfrau  }							% Angabe über den/die Autor(en)
%\date{\today}										% Liefert das aktuelle Datum bei Dokumenterstellung
% maketitel											% Aufruf zum Erstellen und Einbinden der Titelseite
%%%
%
%%%%%%%%%%%%%%%%%%%%%%%%%%%%%%%%%%%% T I T E L S E I T E %%%%%%%%%%%%%%%%%%%%%%%%%%%%%%%%%%%%%%%
% Beginn der ersten Seite im Dokument
%
\pagenumbering{Alph}								% Formatierung der Seitennummerierung auf große Buchstaben
%
\include{01_Titel}
%%%
%
%%%%%%%%%%%%%%%%%%%%%%%%%%%%%%%%%%%%%% V O R S P A N N %%%%%%%%%%%%%%%%%%%%%%%%%%%%%%%%%%%%%%%%%
%
\clearpage \setcounter{page}{-1}						% Start der Seitennummerierung, da Titelseite ohne Seitenzahl
%
\chapter*{Eidesstattliche Erklärung}
\label{chapter_erklaerung}
\thispagestyle{empty}

\vfill

Tim Nieter\\
Rudower Straße 95\\
12351 Berlin

\vspace{\baselineskip}

Hiermit versichere ich, dass ich die von mir vorgelegte Arbeit selbstständig verfasst habe, dass ich die verwendeten Quellen und Hilfsmittel
 vollständig angegeben habe und dass ich die Stellen der Arbeit -- einschließlich Tabellen und Abbildungen --, die anderen Werken oder dem Internet im
 Wortlaut oder dem Sinn nach entnommen sind, auf jeden Fall unter Angabe der Quelle als Entlehnung kenntlich gemacht habe.

\vspace{\baselineskip}

Berlin, den {16. März 2015}

\vspace{1cm}

Tim Nieter

\vspace{.5cm}

\underline{~~~~~~~~~~~~~~~~~~~~~~~~~~~~~~~~~~~~~~~~}\\
(Unterschrift)


%
\chapter*{Sperrvermerk}
\label{chapter_sperrvermerk}
\thispagestyle{empty}

Die nachfolgende Bachelorarbeit mit dem Titel \glqq Entwicklung eines vollautomatisierten Embedded-Linux-Systems zur Ansteuerung und Auswertung eines Langzeittests\grqq\ enthält
vertrauliche Daten der Pepperl+Fuchs GmbH. Veröffentlichungen oder Vervielfältigungen der
Arbeit – auch nur auszugsweise – sind ohne ausdrückliche Genehmigung der Pepperl+Fuchs GmbH
nicht gestattet. Die Arbeit ist lediglich den Korrektoren sowie den Mitgliedern
der Prüfungskommission zugänglich zu machen.




\thispagestyle{empty}


{\large \textbf{Tim Nieter}}\\

{\large \textbf{Thema der Bachelorarbeit}}\\
Entwicklung eines vollautomatisierten Embedded-Linux-Systems zur Ansteuerung und Auswertung eines Langzeittests.\\

{\large \textbf{Stichworte}}\\
BeagleBone Black, Qt, MySQL\\

{\large \textbf{Kurzzusammenfassung}}\\
Im Rahmen dieser Arbeit wird eine Embedded-Linux-System unter Verwendung eines Einplatinencomputers zur Ansteuerung und Auswertung eines Langzeittests entwickelt. Des Weiteren wird zur Erfassung von Messdaten ein RS232 Protokoll und ein MySQL Datenbanksystem implementiert werden. Für die Auswertung und Verwaltung der Daten wird eine Ethernet Schnittstelle und ein Webinterface gestaltet.
% 
\clearpage											% Wechsel auf die nächste freie Seite
\pagenumbering{Roman}								% Seitenzahlen auf große römische Zahlen setzen
\setcounter{page}{0}								% Fortfahren mit Seitenzahl des letzten Abschnitts
%%%
%
%%%%%%%%%%%%%%%%%%%%%%%%%%%%%%%%% V E R Z E I C H N I S S E %%%%%%%%%%%%%%%%%%%%%%%%%%%%%%%%%%%%%
%
%%% Anfang des Inhaltsverzeichnisses
\pdfbookmark{Inhaltsverzeichnis}{toc}				% Bookmark zum Inhaltsverzeichnis
\tableofcontents
\clearpage
%
%%% Anfang des Abbildungsverzeichnisses
\listoffigures
\protect \addcontentsline{toc}{chapter}{Abbildungsverzeichnis}
\clearpage
%
%%% Anfang des Tabellensverzeichnisses
\listoftables
\protect \addcontentsline{toc}{chapter}{Tabellenverzeichnis}
\clearpage
%
%%% Anfang des Abkürzungsverzeichnis
\phantomsection \addcontentsline{toc}{chapter}{Abkürzungsverzeichnis}
\renewcommand\refname{Abkürzungsverzeichnis} \chapter*{Abkürzungsverzeichnis}
%\renewcommand*\bflabel[1]{\textbf{#1}\hfill}
%
% Abkürzungsverzeichnis

\begin{acronym}[MySQL]
\acro{SQL}{Structured Query Language}
\acro{DB}{Datenbank}
\acro{DBMS}{Datenbankmanagementsystem}
\acro{DBS}{Datenbanksystem}
\acro{ER}{Entity-Relationship}
\acro{ERM}{Entity-Relationship-Modell}
\acro{DUT}{Device Under Test}
\acrodefplural{DUT}[DUTs]{Devices Under Test}
\acro{GUI}{Graphical User Interface}
\acro{IDE}{Integrated Developement Environment}
\acro{SDK}{Software Developement Kit}
\acro{LED}{Light-Emitting Diode}
\acro{ADC}{Analog-Digital-Converter}
\acro{LTT}{Long Term Test}
\acro{UART}{Universal Asynchronous Receiver Transmitter}
\acro{BBB}{BeagleBone Black}
\end{acronym}

%
\clearpage
%
%%% Glossarverzeichnis
%\glsaddall
\printglossary[style=altlisthypergroup]
\clearpage
%%%
%
%%% Änderungshistorie
%\input{basix/history}
%%%
%
%%%%%%%%%%%%%%%%%%%%%%%% S E I T E N L A Y O U T  E I N S T E L L E N %%%%%%%%%%%%%%%%%%%%%%%%%%%%%
%
\clearpage \pagenumbering{arabic}					% Seitenzahlen zurücksetzen und Formatierung auf Arabisch
%
\pagestyle{plain}									% Eigenen Seitenstil aktivieren
%
\renewcommand{\chaptermark}[1]{\markboth{\uppercase{\thechapter\ #1}}{}}
\renewcommand{\sectionmark}[1]{\markright{\thesection\ #1}}
\lhead{}
\chead{\leftmark}
\rhead{}
\lfoot{\slshape \nouppercase \rightmark}
\cfoot{}
\rfoot{\bfseries \thepage}
\renewcommand{\headrulewidth}{0.4pt}
\renewcommand{\footrulewidth}{0.4pt}
%%%
%
%%%%%%%%%%%%%%%%%%%% K A P I T E L Ü B E R S C H R I F T  A N P A S S E N %%%%%%%%%%%%%%%%%%%%%%%%%
%###
\titleformat{\part}[display]{\centering\normalfont\fontsize{36}{48}\selectfont\bfseries}{\MakeUppercase{\partname} \thepart}{40pt}{\Huge}
\titlespacing*{\part}{0pt}{50pt}{20pt}
%
% Redefinition con \@currentlabelname fuer \part aufgrund des \titleformat
%
\makeatletter
\let\titlesec@part\part
\renewcommand{\part}{\@ifstar\part@star\part@nostar}
\def\part@star#1{\NR@gettitle{#1}\titlesec@part*{#1}}
\def\part@nostar{\@ifnextchar[\part@nostar@opt\part@nostar@nopt}
\def\part@nostar@nopt#1{\NR@gettitle{#1}\titlesec@part{#1}}
\def\part@nostar@opt[#1]#2{\NR@gettitle{#2}\titlesec@part[#1]{#2}}
\makeatother
%
%\titleformat{\chapter}[frame]{\normalfont}{\filright\large\enspace KAPITEL \thechapter\enspace}{10pt}{\Huge\bfseries\filcenter}
 %\titlespacing{überschriftenklasse}{linker einzug}{platz oberhalb}{platz unterhalb}[rechter einzug]
%\titlespacing*{\chapter}{0pt}{-\topskip}{40pt}
%###
%%
%
%%%%%%%%%%%%%%%%%%%%%%%%%%%%%%%%%%%%%%%%%%%%%%%%%%%%%%%%%%%%%%%%%%%%%%%%%%%%%%%%%%%%%%%%%%%%%%%%%%%%%%%%%%%%%%%%%%%%
%%%%%%%%%%%%%%%%%%%%%%%%%%%%%%%%%%%% BEGINN der wissenschaftlichen Arbeit %%%%%%%%%%%%%%%%%%%%%%%%%%%%%%%%%%%%%%%%%%
%%%%%%%%%%%%%%%%%%%%%%%%%%%%%%%%%%%%%%%%%%%%%%%%%%%%%%%%%%%%%%%%%%%%%%%%%%%%%%%%%%%%%%%%%%%%%%%%%%%%%%%%%%%%%%%%%%%%
%
%%
%%%

\chapter{Einleitung}
\label{chapter_einleitung}
In diesem ersten Kapitel wird auf die Motivationen und die Zielsetzung dieser Arbeit eingegangen.

\section{Motivation}
Wann immer Systeme in der realen Welt eingesetzt werden, sollen sie so zuverlässig, fehlerfrei und vorhersehbar wie möglich arbeiten. Gerade wenn diese Systeme an kritischen Punkten zum Einsatz kommen und über lange Zeiträume agieren, sind diese Eigenschaften besonders wichtig.\\
Um diesen Anforderungen gerecht zu werden, müssen alle Bauelemente eines solchen Systems diese Vorgaben erfüllen, denn die Zuverlässigkeit ist immer abhängig vom schwächsten Glied.
Bei der Entwicklung eines Systems ist es somit entscheidend, alle Bauelemente vorher anhand der gegebenen Umstände zu qualifizieren. Dafür müssen die Grenzen ausgelotet werden, in denen sie zuverlässig betrieben werden können.

Eine dieser Grenzen ist die altersbedingte Änderung der Betriebsparameter, die sogenannte Degradation.
Um das Degradationsverhalten eines Bauelementes bestimmen zu können, muss es über lange Zeiträume unter erschwerten Bedingungen betrieben und ausgewertet. Nur wenn ein Bauelement ein für die Anwendung akzeptables Degradationsverhalten aufweist, ist es für den Einsatz im Gesamtsystem geeignet.\\
Da ein hoher Einsatz von Ressourcen nötig ist, um jedes Bauelement individuell zu Prüfen, existieren automatisierte Teststände mit deren Hilfe der Aufwand minimiert werden soll.
 

\section{Szenario und Zielsetzung}
Ein Unternehmen stellt verschiedene optoelektronischen Sensoren her. Zur Sicherstellung der Zuverlässigkeit der verwendeten \ac{LED} sollen diese mittels eines automatisierten Teststandes bezüglich ihre Degradationsverhaltens überprüft werden.\\
Aus diesem Grund soll im Rahmen dieser Arbeit ein solcher Teststand entwickelt werden. Das beinhaltet die Datenerfassung, die Datenauswertung und die Überwachung des Systems. 
\include{part/02_theorie}
\include{part/03_analyse_konzeptentwicklung}
%%%
%%
%
%%%%%%%%%%%%%%%%%%%%%%%%%%%%%%%%%%%%%%%%%%%%%%%%%%%%%%%%%%%%%%%%%%%%%%%%%%%%%%%%%%%%%%%%%%%%%%%%%%%%%%%%%%%%%%%%%%%%
%%%%%%%%%%%%%%%%%%%%%%%%%%%%%%%%%%%% ENDE der wissenschaftlichen Arbeit %%%%%%%%%%%%%%%%%%%%%%%%%%%%%%%%%%%%%%%%%%%%
%%%%%%%%%%%%%%%%%%%%%%%%%%%%%%%%%%%%%%%%%%%%%%%%%%%%%%%%%%%%%%%%%%%%%%%%%%%%%%%%%%%%%%%%%%%%%%%%%%%%%%%%%%%%%%%%%%%%
%
%%
%%%
%\protect \addtocontents{toc}{\protect\newpage}  	% Seitenumbruch im Inhaltsverzeichnis
\clearpage
%
%%% Abschlussbetrachtung / Ausblick
\include{part/04_abschlussbetrachtung}
%
\clearpage \pagenumbering{roman}
%
%\pagestyle{plain}									% Seitenlayout (Standard) mit zentrierter Seitennummerierung				!!! ÜBERFLÜSSIG???!!!
%
%### Kapiteltitel auf Standardlayout zurücksetzen
\titleformat{\chapter}[display]{\normalfont\huge\bfseries}{\chaptertitlename\ \thechapter}{20pt}{\Huge}
%###%
%
%%%%%%%%%%%%%%%%%% L I T E R A T U R V E R Z E I C H N I S & A N H A N G %%%%%%%%%%%%%%%%%%%%%%%%%%
%
\part{Literatur und Anhang}
\label{part_anhang}
\interlinepenalty = 10000
%
%%% Literaturverzeichnis, mit BibTeX
%
\phantomsection \addcontentsline{toc}{chapter}{Literaturverzeichnis}
%\nocite{*} 										% auch die nicht verwendeten bibtex-Einträge einblenden
\bibliography{bibtex/bibliografie}
%
\clearpage
\include{appendix/appendix}
%\include{part/05_bib_appendix}
%%%
%
% Schmutzblatt (leere Seite am Ende)
%
\newpage                                    
\pagestyle{empty}
\begin{figure}[H]
\centering
%\includegraphics[width=0.9\textwidth]{pix/general/leer.png}
\end{figure}
%
\end{document}
%
%%% CODE - ARCHIV %%%
%\let\origitemize\itemize							% Zeilenabstand in \itemize-Umgebung global verringern
%\def\itemize{\origitemize\itemsep-7pt}
%