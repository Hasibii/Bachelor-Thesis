
\chapter{Einleitung}
\label{chapter_einleitung}
Zur Einleitung dieser Arbeit wird in diesem Kapitel zunächst auf die Motivationen, die zur Erstellung dieser Arbeit geführt haben, eingegangen. Anschließend wird ein Überblick über den Aufbau der Arbeit vermittelt.

\section{Motivation}
Wann immer Systeme in der realen Welt eingesetzt werden, sollen sie so zuverlässig, fehlerfrei und vorhersehbar wie möglich arbeiten. Gerade wenn diese Systeme an kritischen Punkten zum Einsatz kommen und über lange Zeiträume agieren, sind diese Eigenschaften besonders wichtig.\\
Um diesen Anforderungen gerecht zu werden, müssen alle Bauelemente eines solchen Systems diese Vorgaben erfüllen, denn die Zuverlässigkeit ist immer abhängig vom schwächsten Glied.
Bei der Entwicklung eines Systems ist es somit entscheidend, alle Bauelemente vorher anhand der gegebenen Umstände zu qualifizieren. Dafür müssen die Grenzen ausgelotet werden, in denen sie zuverlässig betrieben werden können.

Eine dieser Grenzen ist die altersbedingte Änderung der Betriebsparameter, die sogenannte Degradation.
Um das Degradationsverhalten eines Bauelementes bestimmen zu können, muss es über lange Zeiträume unter erschwerten Bedingungen betrieben und ausgewertet. Nur wenn ein Bauelement ein für die Anwendung akzeptables Degradationsverhalten aufweist, ist es für den Einsatz im Gesamtsystem geeignet.\\
Da ein hoher Einsatz von Ressourcen nötig ist, um jedes Bauelement individuell zu Prüfen, existieren automatisierte Teststände mit deren Hilfe der Aufwand minimiert werden soll.
 
\section{Aufbau der Arbeit}

Im Rahmen dieser Arbeit wird ein Embedded-Linux-System zur Ansteuerung und Auswertung eines Langzeittests zu entwickeln. Dazu wird zunächst in Kapitel \ref{chapter_Grundlagen} auf einige Grundlagen eingegangen. Darunter fallen die Begrifflichkeiten \textit{Embbedded-Linux-System} und \textit{Degradation}, sowie das Konzept einer Datenbank und der C++ Klassenbibliothek Qt.\\
Anschließend werden in Kapitel \ref{chapter_Anforderungsanalyse} die Akteure des Systems analysiert und im Anschluss die Anforderungen an das System definiert.\\
In Kapitel \ref{chapter_Design} wird das Design der Hard- und Softwarekomponenten erläutert. Außerdem wird ein Übertragungsprotokoll für die RS232 Schnittstelle entworfen.\\
Im Anschluss befasst sich Kapitel \ref{chapter_Implementierung} mit der Implementierung des Designs in das System.\\
Um die Ergebnisse zu validieren, wird in Kapitel \ref{chapter_Testen_und_Validieren} auf einige Tests zur Validierung der Ergebnisse eingegangen.\\
Zum Abschluss erfolgt in Kapitel \ref{chapter_FazitUndAusblick} die Reflektion der Lösungen und ein Ausblick für die Zukunft.