
\chapter{Einleitung}
\label{chapter_einleitung}
In diesem ersten Kapitel wird auf die Motivationen und die Zielsetzung dieser Arbeit eingegangen.

\section{Motivation}
Wann immer Systeme in der realen Welt eingesetzt werden, sollen sie so zuverlässig, fehlerfrei und vorhersehbar wie möglich arbeiten. Gerade wenn diese Systeme an kritischen Punkten zum Einsatz kommen und über lange Zeiträume agieren, sind diese Eigenschaften besonders wichtig.\\
Um diesen Anforderungen gerecht zu werden, müssen alle Bauelemente eines solchen Systems diese Vorgaben erfüllen, denn die Zuverlässigkeit ist immer abhängig vom schwächsten Glied.
Bei der Entwicklung eines Systems ist es somit entscheidend, alle Bauelemente vorher anhand der gegebenen Umstände zu qualifizieren. Dafür müssen die Grenzen ausgelotet werden, in denen sie zuverlässig betrieben werden können.

Eine dieser Grenzen ist die altersbedingte Änderung der Betriebsparameter, die sogenannte Degradation.
Um das Degradationsverhalten eines Bauelementes bestimmen zu können, muss es über lange Zeiträume unter erschwerten Bedingungen betrieben und ausgewertet. Nur wenn ein Bauelement ein für die Anwendung akzeptables Degradationsverhalten aufweist, ist es für den Einsatz im Gesamtsystem geeignet.\\
Da ein hoher Einsatz von Ressourcen nötig ist, um jedes Bauelement individuell zu Prüfen, existieren automatisierte Teststände mit deren Hilfe der Aufwand minimiert werden soll.
 

\section{Szenario und Zielsetzung}
Ein Unternehmen stellt verschiedene optoelektronischen Sensoren her. Zur Sicherstellung der Zuverlässigkeit der verwendeten \ac{LED} sollen diese mittels eines automatisierten Teststandes bezüglich ihre Degradationsverhaltens überprüft werden.\\
Aus diesem Grund soll im Rahmen dieser Arbeit ein solcher Teststand entwickelt werden. Das beinhaltet die Datenerfassung, die Datenauswertung und die Überwachung des Systems.