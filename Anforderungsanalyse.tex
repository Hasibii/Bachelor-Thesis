
\chapter{Anforderungsanalyse}
\label{chapter_Anforderungsanalyse}
Gegeben ist ein existierender Degradationsteststand. Die Bedienung des bestehenden Systems bringt jedoch ein hohes Maß an individuellen Anpassungen, einen großen zeitlichen Aufwand, geringe Benutzerfreundlichkeit und Transparenz mit sich. Aus diesen Gründen soll es hingegen der Benutzerfreundlichkeit, der Effizienz und der räumlichen Ausdehnung optimiert werden. Dazu ist es nötig das vorhandene System von Grund auf zu überarbeiten.\\

Folgende Anforderungen werden dabei an das neue System gestellt:
\begin{itemize}
\item Individuelle Parametrierung der \acp{DUT}
\item Automatische Erfassung der Messdaten
\item Auswertung der Messdaten via Fernzugriff
\end{itemize}
\ \\
\textbf{Individuelle Parametrierung der \acp{DUT}}\\
Immer 64 \acp{DUT} vom selben Typ befinden sich auf einem Testboard.

\textbf{Autmatische Erfassung der Messdaten}\\
Die Messdaten der \acp{DUT} sollen zyklisch erfasst werden. Dabei sollen die Intervalle zwischen den Messungen konfigurierbar sein. 
