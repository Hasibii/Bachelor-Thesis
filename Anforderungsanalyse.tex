
\chapter{Anforderungsanalyse}
\label{chapter_Anforderungsanalyse}
Gegeben ist ein existierender Degradationsteststand. Die Bedienung des bestehenden Systems bringt jedoch ein hohes Maß an individuellen Anpassungen, einen großen zeitlichen Aufwand, geringe Benutzerfreundlichkeit und Transparenz mit sich. Aus diesen Gründen soll es hingegen der Benutzerfreundlichkeit, der Effizienz und der räumlichen Ausdehnung optimiert werden. Dazu ist es nötig das vorhandene System von Grund auf zu überarbeiten.\\

Folgende Anforderungen werden dabei an das neue System gestellt:
\begin{itemize}
\item Individuelle Parametrierung der \acp{DUT}
\item Automatische Erfassung der Messdaten
\item Auswertung der Messdaten via Fernzugriff
\end{itemize}
\ \\
\textbf{Individuelle Parametrierung der \acp{DUT}}\\
Immer 64 \acp{DUT} vom selben Typ befinden sich auf einem Testboard. Dabei sollen verschiedene Parameter für die \acp{DUT} berücksichtigt werden. \\
Es müssen die Intervalle zwischen den Messungen in mehreren Stufen einstellbar sein. Das erfolgt durch die Bestimmung von Intervallen und von Zeiträumen in denen diese Intervalle befolgt werden. Außerdem DAC.



\textbf{Automatische Erfassung der Messdaten}\\
Die Messdaten der \acp{DUT} sollen zyklisch erfasst werden. Dabei sollen, wie bereits erwähnt, die Intervalle zwischen den Messungen konfigurierbar sein.\\
Für den einfachen Zugriff auf die Daten, sollen diese in einer \ac{SQL} Datenbank abgelegt werden. Dafür ist eine Kommunikationsschnittstelle zwischen der Datenbank und den Testboards erforderlich. Durch gegebene Hardwarekonfigurationen soll RS232 dafür verwendet werden.

\textbf{Auswertung der Messdaten via Fernzugriff}\\
Zur Auswertung der Messdaten, soll es möglich sein, von einem PC-Arbeitsplatz aus eine Verbindung zu den Messdaten aufzubauen. Außerdem sollen die Messdaten ausgewertet und dargestellt werden.