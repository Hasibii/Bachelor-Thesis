\chapter{Implementieruung}
\label{chapter_Implementierung}

\section{Hardware}

\section{Software}


Das Softwaredesign teilt sich in einen sequentiellen Prozess für die Abfrage und Speicherung der Messwerte, sowie einen Event gesteuerten Prozess für die \ac{GUI} und die externe Kommunikation für die Fernzugriffe. 
Die beiden Programmteile können unabhängig von einander agieren und kommunizieren ausschließlich über Signale und Slots (siehe \ref{QtSignaleSlots}). Durch diese Kapselung ist es möglich die beiden Programmteile durch andere Lösungen auszutauschen, welche lediglich die selben Signale und Slots unterstützen müssen.\ 



entwicklungsumgebung toolchain compiler linaro debugger usw. 

ssh terminal

backup und restore ?

QT crosscompile?

tcp server event gestuert

rs232 timeouts messdatenerfassung

datenbank index zur performanceverbesserung

LCD display energiesparmodus, wecken durch geste

logrotate für logfiles

hw clock aktualisierung bei ntp verbindung

start up script zum starten , registriert uart4 und rtc cape

stromverbrauch display, wlan brauch netzteil

Captive Portal
\subsection{Anwendung zur Messdatenauswertung}