\section{RS232 Protokoll}
\label{section_RS232_Protokoll}
Das Protokoll für die Kommunikation über die RS232 Schnittstelle dient zum Austausch von Informationen innerhalb des Systems zwischen den Akteuren. \\

Folgende Kriterien sollen dabei erfüllt werden:
\begin{itemize}
\item Adressierung individueller Kommunikationspartner
\item Senden verschiedener Befehle
\item Variable Größe der Daten
\item Sicherstellung der Validität der Übertragung
\item Erweiterbar
\end{itemize}



Um eine hohe Zuverlässigkeit gewährleisten zu können, wird eine geringe Bautrate von 2400 verwendet. Da nur geringe Datenmengen in großen Abständen auftreten, vermindert dass die Anfälligkeit der Übertragung gegenüber äußeren Einflüssen ohne dabei die Performance merkbar zu beeinflussen.\\
Des Weiteren wird ein Paritätsbit für eine Paritätsprüfung eingesetzt. Dieses Bit gibt an, ob die Anzahl der Einsen des Datenblocks gerade oder ungerade ist. Der Wert, des verwendeten Even-Paritätsbits ist 1, wenn die Summe der Einsen gerade ist und 0 wenn sie ungerade ist. Dadurch können grobe Fehler bei der Übertragung ausgemacht und korrigiert werden.\\

\subsection{Aufbau}
\begin{table}[H]
\begin{center}
\begin{tabularx}{\textwidth}{|c|c|c|c|X|X|}\hline
 1. Byte & 2. Byte & 3. Byte & 4. Byte & 5. Byte und folgend & Letztes Byte\\ \hline
  Adresse \& R/W & Länge & Befehl & Unterbefehl & Nutzdaten & Checksumme\\ \hline
\end{tabularx}
\caption{Übertragungsrahmen}
\label{table_Frame}
\end{center}
\end{table}

Ein Rahmen des Protokolls besteht aus 4 Steuerbytes, 1 Checksummenbyte und maximal 30 Bytes an Nutzdaten. Durch diesen Aufbau können die Anforderungen erfüllt werden. Im folgenden Abschnitt wird auf die Zusammensetzung und die Funktion der einzelnen Bytes eingegangen.\\



\newpage
\textbf{1. Byte: Adresse \& Read/Write}

\begin{table}[H]
\begin{center}
\begin{tabularx}{\textwidth}{|X|X|X|X|X|X|X|X|}\hline
 7. Bit & 6. Bit & 5. Bit & 4. Bit & 3. Bit & 2. Bit & 1. Bit & 0. Bit\\ \hline
 R/W & Addr6 & Addr5 & Addr4 & Addr3 & Addr2 & Addr1 & Addr0\\ \hline
\end{tabularx}
\caption{1. Byte: Adresse \& Read/Write}
\label{table_1Byte}
\end{center}
\end{table}

Das erste Byte des Übertragungsrahmens setzt sich aus 7 Adressbits und einem Lese-/Schreibbit zusammen. Die ersten 7 Bit (Addr0 - Addr6) bilden die Adresse des anzusteuernden Empfängers. Daraus ergibt sich ein Adressraum von möglichen 128 Adressen, wobei Adresse 0 für neue Mess-Clients zur einmaligen Anmeldung im System reserviert ist (siehe Abschnitt \ref{section_messclientverwaltung}).\\
Das höchste Bit ist das Lese-/Schreibbit. Mithilfe dieses Bits wird unterschieden, ob ein Befehl als Lese- oder Schreibzugriff interpretiert werden soll.

\begin{table}[H]
\begin{center}
\begin{tabular}{|l|l|}\hline
 R/W Bit & Beschreibung \\ \hline
 0 & Die Sender möchte schreiben \\ \hline
 1 & Die Sender möchte lesen \\ \hline
\end{tabular}
\caption{Read/Write}
\label{table_RW}
\end{center}
\end{table}

Ein Lesezugriff stellt dabei eine Anfragen dar. Dass heißt, dass keine Nutzdaten übertragen werden und eine Antwort des Empfängers mit den angeforderten Daten erwartet wird. Bei einem Schreibzugriff hingegen, werden immer Nutzdaten übertragen.\\


\textbf{2. Byte: Rahmenlänge}

\begin{table}[H]
\begin{center}
\begin{tabularx}{\textwidth}{|X|X|X|X|X|X|X|X|}\hline
 7. Bit & 6. Bit & 5. Bit & 4. Bit & 3. Bit & 2. Bit & 1. Bit & 0. Bit\\ \hline
 Len7 & Len6 & Len5 & Len4 & Len3 & Len2 & Len1 & Len0\\ \hline
\end{tabularx}
\caption{2. Byte: Rahmenlänge}
\label{table_2Byte}
\end{center}
\end{table}

Das zweite Byte gibt die Länge des gesamten Rahmens inklusive Steuerbytes, Nutzdaten und Checksumme an. Die minimale Rahmenlänge beträgt 5 Bytes. Dabei handelt es sich um eine Übertragung ohne Nutzdaten und es werden lediglich die 4 Steuerbytes und das Byte für die Checksumme übertragen. Dies geschieht beispielsweise bei einer Leseanfrage.\\
Die maximale Rahmenlänge beträgt 35 Byte. Hierbei werden zusätzlich zu den 4 Steuerbytes und dem Byte für die Checksumme auch die maximale Nutzlast von 30 Byte übertragen. Dieser Fall kann beispielsweise bei Schreibzugriffen auftreten.\\
Anhand der Länge kann eine erste Prüfung der Validität eines Rahmens durchgeführt werden. Sollte die Zahl der empfangenen Bytes sich von der Rahmenlänge unterscheiden, kann von einem ungültigen Rahmen ausgegangen werden.\\



\textbf{3. Byte: Befehl}

\begin{table}[H]
\begin{center}
\begin{tabularx}{\textwidth}{|X|X|X|X|X|X|X|X|}\hline
 7. Bit & 6. Bit & 5. Bit & 4. Bit & 3. Bit & 2. Bit & 1. Bit & 0. Bit\\ \hline
 Cmd7 & Cmd6 & Cmd5 & Cmd4 & Cmd3 & Cmd2 & Cmd1 & Cmd0\\ \hline
\end{tabularx}
\caption{3. Byte: Befehl}
\label{table_3Byte}
\end{center}
\end{table}

Das dritte Byte repräsentiert den Befehl. Dieser gibt an, welche Aktion ausgeführt oder welcher Parameter angesprochen wird. Da insgesamt ein Byte für den Befehl zur Verfügung steht, sind bis zu 256 unterschiedliche Befehle möglich.\\


\textbf{4. Byte: Unterbefehl}

\begin{table}[H]
\begin{center}
\begin{tabularx}{\textwidth}{|X|X|X|X|X|X|X|X|}\hline
 7. Bit & 6. Bit & 5. Bit & 4. Bit & 3. Bit & 2. Bit & 1. Bit & 0. Bit\\ \hline
 Scmd7 & Scmd6 & Scmd5 & Scmd4 & Scmd3 & Scmd2 & Scmd1 & Scmd0\\ \hline
\end{tabularx}
\caption{4. Byte: Unterbefehl}
\label{table_4Byte}
\end{center}
\end{table}

Das vierte Byte ist der Unterbefehl. Damit ist es möglich einen Befehl genauer zu spezifizieren. So kann beispielsweise der Befehl zum Auslesen eines Messwertes durch den Unterbefehl genau auf einen der 64 Prüfobjekte präzisiert werden. Im späteren Verlauf wird in Abschnitt \ref{section_BefehleUnterbefehle} näher auf die vorhandenen Befehle und Unterbefehle eingegangen.\\



\textbf{5. Byte und folgend: Nutzdaten}

\begin{table}[H]
\begin{center}
\begin{tabularx}{\textwidth}{|X|X|X|X|X|X|X|X|}\hline
 7. Bit & 6. Bit & 5. Bit & 4. Bit & 3. Bit & 2. Bit & 1. Bit & 0. Bit\\ \hline
 Data7 & Data6 & Data5 & Data4 & Data3 & Data2 & Data1 & Data0\\ \hline
\end{tabularx}
\caption{5. Byte und folgend: Nutzdaten}
\label{table_5Byte}
\end{center}
\end{table}

Das fünfte Byte und die darauf folgenden, tragen die Nutzdaten des Rahmens. Die Größe der Nutzdaten ist variabel und lässt sich von der Rahmenlänge ableiten. Bei 16 Bit Variablen wird immer im Big-Endian-Format übertragen. \\


\newpage

\textbf{Letztes Byte: Checksumme}

\begin{table}[H]
\begin{center}
\begin{tabularx}{\textwidth}{|X|X|X|X|X|X|X|X|}\hline
 7. Bit & 6. Bit & 5. Bit & 4. Bit & 3. Bit & 2. Bit & 1. Bit & 0. Bit\\ \hline
 CKS7 & CKS6 & CKS5 & CKS4 & CKS3 & CKS2 & CKS1 & CKS0\\ \hline
\end{tabularx}
\caption{Letztes Byte: Checksumme}
\label{table_LastByte}
\end{center}
\end{table}

Das letzte Byte ist immer die Checksumme, um sicherzustellen, dass alle Daten komplett und fehlerfrei übertragen wurden.\\
Der Sender bildet dabei die Checksumme mittels einer XOR Verknüpfung aller Bytes eines Übertragungsrahmens. Beim Empfänger werden alle Bytes inklusive Checksumme erneut mit XOR Verknüpft, wobei ohne Fehler immer 0 das Ergebnis sein muss.\\


\textbf{Beispiel}

Sender:
 
Es wird angenommen das folgender Rahmen übertragen werden soll.
\begin{center}
\begin{tabular}{|c|c|c|c|}\hline
  Adresse & Länge & Befehl & Unterbefehl \\ \hline
  1 & 5 & 9 & 0 \\ \hline
\end{tabular}
\end{center}
Aus dem Rahmen wird dann mittels XOR die Checksumme gebildet.

\begin{center}
{\Large$1 \oplus 4 \oplus 9 \oplus 0 = 13$}\\
\end{center}

Anschließend wird die Checksumme als letztes Byte an den Rahmen an gehangen.
\begin{center}
\begin{tabular}{|c|c|c|c|c|}\hline
  Adresse & Länge & Befehl & Unterbefehl & Checksumme \\ \hline
  1 & 5 & 9 & 0 & 13 \\ \hline
\end{tabular}
\end{center}

Empfänger:

Sobald der Empfänger den Rahmen erhält, verknüpft er alle Bytes erneut mittels XOR um die Gültigkeit zu prüfen.

\begin{center}
{\Large $1 \oplus 4 \oplus 9 \oplus 0 \oplus 13 = 0$}\\
\end{center}

Das Ergebnis 0 repräsentiert einen gültigen Rahmen und die Checksummenprüfung war erfolgreich.


\subsection{Befehle und Unterbefehle}
\label{section_BefehleUnterbefehle}
Für die Funktion des Systems sind verschiedene Befehle notwendig. Sie dienen zu Kommunikation zwischen den Akteuren.
Eine Übersicht über die implementierten Befehle der RS232 Schnittstelle findet sich in Tabelle \ref{table_RS232Commands}.\\

\begin{table}[H]
\begin{center}
\begin{tabularx}{\textwidth}{|X|c|X|c|}\hline 
 Befehl & Code & Unterbefehl & Datenbytes \\ \hline \hline
 ADC-Value & 0x00 & MUX-Kanal (0..63) & 2  \\ \hline
 Number Of Pulses & 0x04 & - & 1  \\ \hline
 Pulsewidth and -period & 0x05 & Pulsnummer (1..20) & 4  \\ \hline
 Perform Pulseupdate & 0x06 & - & 0   \\ \hline
 DAC-Value & 0x07 & - & 2 \\ \hline
 Temperature & 0x08 & - & 1  \\ \hline
 LTT Name & 0x09 & - & 1..30  \\ \hline
 Rs232-Address & 0x0A & - & 1 \\ \hline
 Measurement Intervall & 0x0C & Intervallnummer (0..2) & 4 \\ \hline
\end{tabularx}
\caption{RS232 Befehlsliste}
\label{table_RS232Commands}
\end{center}
\end{table}

\textbf{ADC-Value}\\
Liest den Messwert eines Sensors des Mess-Clients. Mit dem Unterbefehl muss einer der 64 Sensoren spezifiziert werden. Ein Messwert hat die Größe von 2 Byte. Dabei wird das höhere Byte zuerst übertragen.\ 

\textbf{Number Of Pulses}\\
Liest oder schreibt die Anzahl der Impulse im Pulsmuster des Mess-Clients. Die maximale Anzahl ist 20 und sie wird als 1 Byte Wert übertragen.\ 

\textbf{Pulsewidth and -period}\\
Liest oder schreibt  ein Pulsbreiten-/Pulsperiodenpaar. Der Unterbefehl bestimmt die Position im Pulsmuster. Sowohl die Pulsbreite als auch die Pulsperiode sind jeweils 2 Byte groß. Die ersten 2 Byte repräsentieren die Pulsbreite und die zweiten 2 Byte die Pulsperiode. Die Übertragung erfolgt im Big-Endian-Format. \ 

\textbf{DAC-Value}\\
Liest oder schreibt den Wert für die optische Leistung der Prüfobjekte. Sie ist 12 Bit groß und hat somit einen Wertebereich von 0 bis 4096. Die Übertragung erfolgt im Big-Endian-Format.\ 

\textbf{Temperatur}\\
Der Befehl ließt den Wert des Temperatursensors. Er ist nur zum Lesen des Temperaturwertes gedacht und kann nicht zum schreiben verwendet werden. Es handelt sich um einen signed 1 Byte Wert. Der Wertebereich geht von 127 bis -127.\ 

\textbf{LTT Name}\\
Liest oder schreibt den Namen des Mess-Clients. Die maximale Zeichenlänge beträgt 30 Zeichen. Das erste Zeichen wird auch zuerst übertragen.\ 

\textbf{RS232-Adresse}\\
Liest oder schreibt die Adresse für die Rs232 Kommunikation des Mess-Clients. Es handelt sich um einen 7 Bit Wert. Der Wertebereich geht von 0 bis 127 was 128 Adressen entspricht.\ 

\textbf{Measurement Intervall}\\
Liest oder schreibt die Konfiguration der Messintervalle. Der Unterbefehl gibt dabei den Zeitraum (1 bis 3) an. Die ersten 2 Byte enthalten das Intervall zwischen den Messungen und die zweiten 2 Byte den Zeitraum. Die Übertragung erfolgt im Big-Endian-Format.\ 

